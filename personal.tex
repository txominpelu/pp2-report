


\documentclass[12pt,a4paper]{article}

\usepackage[utf8]{inputenc}
\usepackage[american]{babel}
\usepackage[T1]{fontenc}
\usepackage{listings}
\usepackage{graphicx}

\title{NoWork\\ Personal Report}
\author{Inigo Mediavilla\\[2em]}
\date\today

\begin{document}
\maketitle

%% Index
%% - Introduction to Rewriting system (what it is, what is useful for)
%% - Justification for the project (what have we gained from doing,
%% how could a user benefit from it)
%%
%% - Global vision of the project (short description of how to
%% describe a language with the system, describing every one of the concepts)
%%
%% - Subsystems (a short description of every part
%% of the project (parsing, typechecking
%% - How does the system work? Explain how the rewriting of a system
%% is implemented by giving a term and explaining what is the term,
%% what is the pattern, what is the metavariable of placeholder
%% -
%% - Problem with rewriting (what subtree should I replace first -
%% show with an example why it matters because it affects the result)
%% - Initial Naive version of rewriting: (1. without fixpoint (Problems)
%% 2. Bottom-up and Top-down 3. Higher flexibility Strategies)
%%
%% - Conclusion

\section{Introduction to rewriting systems}

Programming languages (or other systems based on rewriting rules)
are defined as a set of transformations called
rewriting rules. This rewriting rules are the core of the language and
they define what is the computational power of the language as well as
its expressivity and other properties.

This systems based on rewriting rules have applications that are not
restricted to the well known general paradigm programming languages
for computers. Systems based on this kind of rules have been used for
the analysis of bussiness interactions or for the formalization of
some logics, for example the Heyting-Kolmogorov mapping has shown that
predicate logic can be expressed as a set of rewriting rules.

One of the reasons why this kind of systems are so widely used is that
they have been well studied, they are well understood and a
wide range of tools exist to prove properties on them.

\section{Justification}

Despite the multiple applications of these systems and all the
mathematical tools available to prove properties on them building a
new rewriting system or tuning an existing one for experimentation is
still a really tedious task. The implementation of this kind of
system from scratch requires building a parser for the language, a
type checker, and all the code to find matches for a rule and
transforming the existing term when matches are found. Most of these
steps require a great amount of work even though they're relatively
simple. Besides a great amount of this work could be generalized since
the difference between two of these systems are reduced only to the
definition of the rules.

This is the purpose of Nowork, to provide the generic part in the
implementation of this kind of system allowing users to define new
systems by only defining the rules that define them.

%% It also allows to prove some properties on them and further work
%% on nowork could also facilitate the automatic verification of
%% properties for these systems.

\section{ Pedagogical value }

\section{ NoWork - Rewriting System }

NoWork users can define their own rewriting system without having to
write their own parser, typechecker and whithout having to code the
reduction rules of the interpreter of the programming language. To do
that NoWork offers a language where all the rules can be defined as
well as all the types, constants and operations available in the
language.

Rules are defined as a pattern  and a transformation. The pattern
describes the model that NoWork will try to find in a given term
to replace it by the transformation specified by the rule.

\begin{figure}[!h]
\begin{verbatim}

rule = pattern + transformation

e.g App(Lambda(x?, y?), z? ) => Subst(x?, z?, y?) // Substituting x by
z in y
\end{verbatim}
\caption{Structure of a pattern}
\end{figure}

On top of that NoWork allows a simple syntax for defining Constants.

\begin{figure}[!h]
\begin{verbatim}

constant Nil : forall(A).List<A>
\end{verbatim}
\caption{Definition of a constant}
\end{figure}


It also allows to define Operators that are not like the higher order functions
that we know in functional programming languages since all the
reduction in NoWork is specified in the reduction rules. Operators
just give the name and the type of a reduction. Usually in the
definition of a language every operator will have a rewriting rule
that will take a subtree whose root has the same name as the operator
and if the language is well formed the transformation will conform to
the rule defined in the operator.


\begin{figure}[!h]
\begin{verbatim}

operator Cons : forall(A).A * List<A> -> List<A>

\end{verbatim}
\caption{Definition of an operator}
\end{figure}

Kinds allow to define the constant types and also the polymorphic types when
their definition depends on a type variable.

\begin{figure}[!h]
\begin{verbatim}

kind Bool : type // Constant type
kind List : type -> type // Polymorphic type
\end{verbatim}
\caption{Definition of kinds}
\end{figure}

Therefore, the rewriting rules provide the semantics of the language
while the type annotations provided in the constants, the operators as well
as the kinds allow the user to define the typing rules. All together
this language helps the user concisely define a programming language
declaratively without having to implement a parser, a typechecker and
an interpreter.

\section{Global vision of the system}

We can consider that programs written in NoWork use two separate languages:

Definition of the language. Written in NoWork's language. Contains the
constants, operators, kinds and rules as they have been described
previously. This definition must conform to the grammar and type
checking rules of NoWork's language to avoid the definition of
malformed languages.

Language Terms. Once a language has been defined with NoWork terms of
that language can be written as trees. This trees should conform to the
typing rules that have been defined for the language that are
independent of NoWork's language rules.

In addition NoWork provides a mini-language with a list of commands that are parsed by
NoWork's toplevel and that allow the user to interact with the
program. When the toplevel receives command it executes a list of
steps that can be generalized to:

\subsection{Process}

\begin{figure}[!h]
\begin{verbatim}
Parsing + Type Checking  + Rewriting (includes pattern matching + transformation)
\end{verbatim}
\caption{ Phases of the evaluation of a command }
\end{figure}

Parsing: Read file with the definition an convert it into an ast that
can contain a definition of a language or an interaction with a term
of the language

Typechecking: If the command is a definition of the language NoWork's
Language typing rules are verified. Otherwise the command is an interaction with a
term and the term needs to be typed checked against the definition of
the language contained in the environment.

Rewriting: This phase occurs only when the command is not a definition
of the language. In this phase the term is explored with a
given strategy looking for matches of the patterns described in the
rewriting rules of the language. When a match is found the
transformation is applied to the part of the term where the match
was found.

\section{Rewriting}


\section{Conclusion}

What have I learned, What are the disadvantages and advantages of the
solution proposed, how could this implementation could be improved in
the future...

Problem:

Defining a framework for writing nominal rewriting systems:

Why? Because it is useful to play with different variations of a
rewriting systems to understand the differences in expressivity, ``puissance'',
congruence, and normalisation properties

(Describe every one of those properties)

This kind of tool with the right support for traces and the right level of explanation in every
operation can be a powerful learning tool.

From a pedagogical point of view the implementation of nominal
workbench has provided me with a deeper understanding of rewriting,
considering this concept in Peano Arithmetic, or CCS has widened the
vision that I had for this concept mainly reduced to lambda calculus
at the beginning of the project.


Why nominal and non-linera? The additional adjectives of non-linear and nominal 


\end{document}
